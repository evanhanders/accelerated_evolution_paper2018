%\documentclass[iop]{emulateapj}
\documentclass[aps, pre, onecolumn, nofootinbib, notitlepage, groupedaddress, amsfonts, amssymb, amsmath, longbibliography]{revtex4-1}
\usepackage{graphicx}
\usepackage{hyperref}
\usepackage{xcolor}
\hypersetup{
    colorlinks,
    linkcolor={red!50!black},
    citecolor={blue!50!black},
    urlcolor={blue!80!black}
}
\usepackage{bm}
\usepackage{natbib}
\usepackage{longtable}
\LTcapwidth=0.87\textwidth

\newcommand{\Div}[1]{\ensuremath{\nabla\cdot\left( #1\right)}}
\newcommand{\angles}[1]{\ensuremath{\left\langle #1 \right\rangle}}
\newcommand{\grad}{\ensuremath{\nabla}}
\newcommand{\RB}{Rayleigh-B\'{e}nard }
\newcommand{\stressT}{\ensuremath{\bm{\bar{\bar{\Pi}}}}}
\newcommand{\lilstressT}{\ensuremath{\bm{\bar{\bar{\sigma}}}}}
\newcommand{\nrho}{\ensuremath{n_{\rho}}}
\newcommand{\approptoinn}[2]{\mathrel{\vcenter{
	\offinterlineskip\halign{\hfil$##$\cr
	#1\propto\cr\noalign{\kern2pt}#1\sim\cr\noalign{\kern-2pt}}}}}

\newcommand{\appropto}{\mathpalette\approptoinn\relax}

\newcommand\mnras{{MNRAS}}%

\begin{document}
\author{Evan H. Anders}
\affiliation{Dept. Astrophysical \& Planetary Sciences, University of Colorado -- Boulder, Boulder, CO 80309, USA}
\affiliation{Laboratory for Atmospheric and Space Physics, Boulder, CO 80303, USA}
\author{Benjamin P. Brown}
\affiliation{Dept. Astrophysical \& Planetary Sciences, University of Colorado -- Boulder, Boulder, CO 80309, USA}
\affiliation{Laboratory for Atmospheric and Space Physics, Boulder, CO 80303, USA}
\title{BVP Paper}

\begin{abstract}
WOW this is a really long sentence check out this abstract I'll just keep writing words to make this at least
one line long so we know what the formatting looks like, ok?
\end{abstract}
\maketitle


\section{Introduction}
\label{sec:intro}

\section{Experiment} 
\label{sec:experiment}

\section{Results \& Discussion}
\label{sec:results}




\begin{acknowledgments}
EHA acknowledges the support of the University of Colorado's George 
Ellery Hale Graduate Student Fellowship.
This work was additionally supported by  NASA LWS grant number NNX16AC92G.  
Computations were conducted 
with support by the NASA High End Computing (HEC) Program through the NASA 
Advanced Supercomputing (NAS) Division at Ames Research Center on Pleiades
with allocations GID s1647 and GID g26133.
\end{acknowledgments}


\appendix
\section{Process for setting enthalpy flux}


%\bibliography{../biblio.bib}
\end{document}
