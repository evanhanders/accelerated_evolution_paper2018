%\documentclass[iop]{emulateapj}
\documentclass[aps, pre, onecolumn, nofootinbib, notitlepage, groupedaddress, amsfonts, amssymb, amsmath, longbibliography]{revtex4-1}
\usepackage{tabularx}
\usepackage{graphicx}
\usepackage{hyperref}
\usepackage{xcolor}
\hypersetup{
    colorlinks,
    linkcolor={red!50!black},
    citecolor={blue!50!black},
    urlcolor={blue!80!black}
}
\usepackage{bm}
\usepackage{natbib}
\usepackage{longtable}
\LTcapwidth=0.87\textwidth

\newcommand{\Div}[1]{\ensuremath{\nabla\cdot\left( #1\right)}}
\newcommand{\DivU}{\ensuremath{\nabla\cdot\bm{u}}}
\newcommand{\angles}[1]{\ensuremath{\left\langle #1 \right\rangle}}
\newcommand{\grad}{\ensuremath{\nabla}}
\newcommand{\RB}{Rayleigh-B\'{e}nard }
\newcommand{\stressT}{\ensuremath{\bm{\bar{\bar{\Pi}}}}}
\newcommand{\lilstressT}{\ensuremath{\bm{\bar{\bar{\sigma}}}}}
\newcommand{\nrho}{\ensuremath{n_{\rho}}}
\newcommand{\approptoinn}[2]{\mathrel{\vcenter{
	\offinterlineskip\halign{\hfil$##$\cr
	#1\propto\cr\noalign{\kern2pt}#1\sim\cr\noalign{\kern-2pt}}}}}

\newcommand{\appropto}{\mathpalette\approptoinn\relax}

\newcommand\mnras{{MNRAS}}%

\begin{document}
\author{Evan H. Anders}
\affiliation{Dept. Astrophysical \& Planetary Sciences, University of Colorado -- Boulder, Boulder, CO 80309, USA}
\affiliation{Laboratory for Atmospheric and Space Physics, Boulder, CO 80303, USA}
\author{Benjamin P. Brown}
\affiliation{Dept. Astrophysical \& Planetary Sciences, University of Colorado -- Boulder, Boulder, CO 80309, USA}
\affiliation{Laboratory for Atmospheric and Space Physics, Boulder, CO 80303, USA}
\author{Jeffrey S. Oishi}
\affiliation{Department of Physics and Astronomy, Bates College, Lewiston, ME 04240, USA}
\title{Accelerated convergence of convective simulations using boundary value problems}

\begin{abstract}
We present a method for using coupling Boundary value problems (BVPs) with Initial value problems (IVPs)
in order to achieve thermally converged convective solutions on dynamical timescales, rather than the
long thermal timescale.  We demonstrate the similarity between the solution reached via BVP and the
solution reached by a long thermal rundown of the IVP, and demonstrate that this method works at a
large range of supercriticalities.  We use this method to achieve converged solutions at high Ra,
and discuss its extension to more complex scenarios, such as stratified, compressible convection.
\end{abstract}
\maketitle

\section{To Do}
\begin{enumerate}
\item Get 3D cases in
\item Continue Improving Figures
\item Write figure captions
\end{enumerate}


\section{Introduction}
\label{sec:intro}
Natural convection occurs in the presence of disparate timescales. Granules on the
solar surface overturn on the order of 10 minutes, whereas deep motions in the Sun are likely at
low Mach number and constrained by the solar rotation rate of $\sim$1 month.  
Both of these dynamical timescales are vastly shorter than the Sun's average timescale of energy transport,
which is the Kelvin-Helmholtz timescale of nearly $3 \times 10^7$ \emph{years} \cite{stix2003}. 
As simulations aim to model natural convection
by increasing into the high-Rayleigh Number (Ra) regime, where diffusive timescales are much
longer than dynamical timescales \cite{Anders&Brown2017}, achieving converged simulations will 
require runs which span a greater number of convective timescales in order to thermally converge.
Furthermore, with increasing Ra and decreasing diffusivities, motions become more turbulent
and require finer grid meshes and shorter timesteps to resolve turbulent motions, increasing the simulation
time required for each overturn timescale.  
These two effects conspire to make achieving thermally converged, high-Ra, astrophysically interesting
simulations an intractable problem using modern numerical tools.

Prior studies of stratified convection in which a convective layer lies between stable layers
have used the knowledge of Mixing Length Theory (MLT) to adjust the initial thermodynamic structure of
the atmosphere to a state which is closer to the adiabat chosen by convection \cite{brandenburg&all2005}.
However, most studies of convection do not contain stable layers above and below the convection
zone, and the presence of hard boundaries and the boundary layers that they form make it difficult to
know the correct evolved adiabat \emph{a priori}.

Here we present a method for using simple boundary value problems (BVPs), 
along with information about the evolved flow fields,
to fast-forward the slow thermal evolution of convecting simulations.  
We run two sets of experiments: one in which we allow convective simulations to evolve for a
full thermal timescale before taking measurements, and another in which we employ a fast-forwarding,
BVP technique which occurs on dynamical timescales. We compare these two sets of simulations to
show the validity of the BVP technique.  Then, we use the BVP technique to run simulations
at high Ra, in the regime where running for thermal timescales becomes computationally intractable.

\section{Experiment}
\label{sec:experiment}
In our study, we adopt the Oberbeck-Boussinesq approximation.  Here, the
fluid has constant kinematic viscosity ($\nu$), thermal diffusivity ($\kappa$), fluid density ($\rho_0$), and coefficient
of thermal expansion ($\alpha$).   The Boussinesq
equations governing the velocity $\bm{u} = u\hat{x} + v\hat{y} + w\hat{z}$, temperature
$T = T_0 + T_1$, and pressure $P$ are \cite{spiegel&veronis1960}
\begin{gather}
\DivU = 0, 
	\label{eqn:dim_incompressible}
\\
\frac{\partial \bm{u}}{\partial t} + \bm{u}\cdot\grad\bm{u} =
-\frac{1}{\rho_0}\grad P - g( 1 - \alpha T_1)\hat{z} + \nu\grad^2\bm{u}, 
	\label{eqn:dim_bouss_momentum}
\\
\frac{\partial T_1}{\partial t} + \bm{u}\cdot\grad(T_0 + T_1) = \kappa\grad^2 T_1,
	\label{eqn:dim_bouss_energy}
\end{gather}
We non-dimensionalize these equations on a length scale of the layer height ($L_z$),
a temperature of the (constant) initial temperature gradient across the layer ($\grad T_0$), and a timescale
of the freefall timescale ($L_z / v_{\text{ff}}$, with $v_{\text{ff}} = \sqrt{\alpha g L_z^2 \grad T_0}$, where $g$ is 
uniform gravitational acceleration in the $-\hat{z}$ direction).
We non-dimensionalize the equations and re-arrange the momentum equation, introducing a reduced pressure
$\varpi \equiv P / \rho_0 + \phi + |\bm{u}|^2 / 2$, where $\phi$ is the gravitational potential, we time evolve
the following equations,
\begin{gather}
\DivU = 0, 
	\label{eqn:incompressible}
\\
\frac{\partial \bm{u}}{\partial t} + \grad \varpi - T_1\hat{z} + \mathcal{R}\grad\times\bm{\omega} = \bm{u}\times\bm{\omega}
	\label{eqn:bouss_momentum}
\\
\frac{\partial T_1}{\partial t} - \mathcal{P}\grad^2 T_1 + w \frac{\partial T_0}{\partial z} = - \bm{u}\cdot\grad T_1,
	\label{eqn:bouss_energy}
\end{gather}
where $\bm{\omega} = \grad \times \bm{u}$ is the vorticity.  
Here the dimensionless control parameters are the Rayleigh and Prandtl numbers,
\begin{equation}
\mathcal{R} \equiv \sqrt{\frac{\text{Pr}}{\text{Ra}}}, \qquad \mathcal{P} \equiv \frac{1}{\sqrt{\text{Pr}\,\text{Ra}}}, \qquad
\text{Ra} = \frac{g \alpha L_z^4 \grad T_0}{\nu\chi} = \frac{(L_z\,v_{\text{ff}})^2}{\nu\chi}, \qquad \text{Pr} = \frac{\nu}{\chi}.
\end{equation}
The dimensionless vertical extent of the domain is $z = [-1/2, 1/2]$, and at the boundaries
we impose no-slip, impenetrable boundary conditions such that $w = u = v = 0$ at $z = \pm 1/2$.
At the lower boundary, we employ a fixed flux condition such that $\partial T_1 / \partial z = 0$
at $z = -1/2$, and we impose a fixed temperature condition $T_1 = 0$ at $z = 1/2$. Both
horizontal directions are periodic, extending over a range $x, y = [0, \Gamma]$, where
the aspect ratio is $\Gamma = 2$.



These equations are expressed with linear terms on the left-hand side and
nonlinear terms on the right-hand side.  Because $P$ is a (whatever it's called) in boussinesq, \RB convection, we are able to
evolve $\varpi$ as a linear variable even though it contains a nonlinear velocity component.  This formulation
of the equations gives us a modest improvement in timestepping speed compared to Eqns. (\ref{eqn:dim_incompressible}-\ref{eqn:dim_bouss_energy}).
have been expressed with the linear terms (which we solve implicitly) on the LHS, and the
nonlinear terms on the RHS. Here, the reduced pressure is $\varpi \equiv P / \rho_0 + \phi + |\bm{u}|^2 / 2$,
where $\phi$ is the gravitational potential and where $P$ is a (I forget the word) that enforces incompressibility
in the Boussinesq system.

The chosen thermal boundary conditions at the upper and lower plates
determine key quantities of the evolved state.
Studies of incompressible, Boussinesq, \RB convection often
employ fixed temperature (Dirichlet) or fixed heat flux
(Neumann) boundary conditions at both plates.  
Dirichlet conditions represent plates of infinite conductivity,
whereas Neumann conditions model plates of finite conductivity.  
In both cases, choosing symmetric boundary conditions maintains overall system symmetry, 
and despite evolving towards different thermal structures, both types of conditions
transport heat in the same manner \cite{johnston&doering2009}.
Studies of convection which aim to model
astrophysical systems such as stars often employ a mixture of these
two types of boundary conditions \cite{hurlburt&all1984, cattaneo&all1991, korre&all2017}.  
The flux at the lower boundary is fixed, modeling
the constant energy generation of the stellar core, 
while the outer boundary condition is held at a fixed temperature,
modeling the surface of a star which must output the energy generated internally.
This setup is a useful model for understanding natural
systems, but simulations which employ these boundary conditions suffer from a long 
thermal relaxation as the atmosphere loses energy and approaches the adiabat chosen by the
Dirichlet condition.  We choose these conditions in part to better understand them, and in
part because these conditions minimize the number of assumptions that must be made in
setting up the boundary value problem.




\subsection{The Boundary Value Problem}
\begin{figure}[b]
\includegraphics[width=\textwidth]{./figs/time_trace.png}
\caption{Traces of system energies vs. time for a long thermal rundown (a) and BVP convergence
(b) are shown for Ra = $1.30 \cdot 10^7$ ($S = 10^4$).  The two plots are scaled such that
one time unit on the x-axis of either run takes up an equivalent amount of space on the paper
to illustrate the time savings of the BVP run.  (c\&d) Flux plots, where black is the sum of the
flux, blue is enthalphy flux and red is condcutive flux. (c) Fluxes are shown for the IVP at an early
time to illustrate the asymmetry of the flux through the system. (d) Fluxes through the converged,
rundown state are compared to the fluxes in the converged BVP solution, where their difference
is shown in (e). \label{fig:time_trace} }
\end{figure}

The prohibitively long thermal timescale required to reach an equilibrium temperature profile
in a Direct Numerical Simulation (DNS) can be skipped by coupling the DNS with a simple Boundary Value Problem
(BVP) solve. Using information about the dynamics of the convection in the atmosphere, it is possible
to skip a large portion of the thermal evolution, see Fig \ref{fig:time_trace}a\&b.

The Boussinesq BVP contains equations of hydrostatic balance and thermal equilibrium,
\begin{gather}
\frac{\partial}{\partial z}\angles{\varpi} = \angles{T_1}\hat{z},
	\label{eqn:bouss_BVP_momentum}
\\
\frac{\partial}{\partial z}\angles{wT_1} = \frac{1}{\text{Pr}\text{Ra}}\frac{\partial^2}{\partial z^2} \angles{T_1},
	\label{eqn:bouss_BVP_energy}
\end{gather}
where $\angles{A}$ represents a time- and horizontally averaged profile of the quantity $A$.  
These
equations arise from taking time- and horizontal- averages of Eqns (\ref{eqn:bouss_momentum}\&\ref{eqn:bouss_energy}),
neglecting terms that vanish due to symmetry, and assuming that the flows are in a stationary state.  
Convective flows
are perturbations around a thermal profile defined by these equations in the proper evolved state.

Under eqns (\ref{eqn:bouss_BVP_momentum}\&\ref{eqn:bouss_BVP_energy}), 
the thermal structure ($\angles{T_1}$, $\angles{\varpi}$) of the atmosphere is fully determined by the specification
of the convective flux, $F_{conv} = \angles{w T_1}$.  If this profile is known, then 
solving for $\angles{T_1}$ and
$\angles{\varpi}$ depends only upon the choice of boundary conditions.

By definition, the profile of $F_{conv}$ is \emph{not} in its time stationary state near the
beginning of the simulation.  In fact, under mixed boundary conditions, as the atmosphere approaches the
isotherm specified by the upper boundary condition, the motions display an asymmetric flux as energy
leaks through the upper boundary condition (Fig. \ref{fig:time_trace}c).  
In order to construct the evolved convective flux from the current fluxes in the atmosphere,
we acknowledge that the evolved solution will be in flux equilibrium, 
carrying the amount of flux specified at the fixed-flux condition throughout the full depth of the atmosphere.  
Thus, the steady-state profile of the convective flux can be approximated as
\begin{equation}
F_{\text{conv, steady}} = F_{\text{bot}}\frac{\angles{wT_1}}{\angles{wT_1 - \kappa \partial_z (T_0 + T_1)}}
= F_{\text{bot}}\frac{\angles{F_{\text{conv, IVP}}}}{\angles{F_{\text{tot, IVP}}}}.
\label{eqn:bouss_BVP_fconv}
\end{equation}
In essence, the construction of this profile assumes that the system appropriately picks out the ratios
\begin{equation}
f_{\text{conv}} = \frac{F_{\text{conv}}}{F_{\text{tot}}}\qquad
f_{\text{cond}} = \frac{F_{\text{cond}}}{F_{\text{tot}}}.
\end{equation}
in the transient state.  Thus, even though the quantity of flux being carried is not correct, the
system is carrying flux convectively where it needs to in the interior, and conductively in the boundaries.

The choice of a fixed-flux boundary condition at the bottom appropriately scales the magnitude of
$F_{\text{conv, steady}}$.  The choice of a fixed-temperature boundary condition at the top of the
atmosphere appropriately sets the isotherm of the convective interior.  While this method can be used
for other choices of thermal boundary conditions (see Discussion), dual fixed temperature conditions
at the upper plate require a more careful handling of $F_{\text{bot}}$, and dual fixed flux boundary
conditions have degenerate solutions for the constant offset of $T_1$.

In general, the BVP solve takes the following form:
\begin{enumerate}
\item Run the convective IVP. Once the convection begins, start taking averages averages of $\angles{wT_1}$
and $\partial_z^2\angles{T_1}$, which determine the fluxes through the system.  Update these averages every
$\Delta t = 0.1$ freefall times.  Once these averages are converged to 1 part in 1000, the BVP is ready to be solved.
\item Construct $F_{\text{conv, steady}}$ from the flux profiles, then use it to solve for $\angles{T_1}$ 
and \angles{\varpi} of the
evolved state.  Adjust the mean profiles in the BVP such that this is their mean profile.
\item Divide the velocities and the $T_1$ fluctuations around the mean profile by 
$\sqrt{\bar{F_{\text{bot}}/F_{\text{tot}}}}$. This lowers the convective flux through the system such that it is,
on average, the convective flux used in the BVP solve.
\item Continue running the IVP for a number of freefall times to allow the velocities and temperature perturbations
to equilibrate to the new mean state.
\end{enumerate}
This procedure seems to work quite well, see Fig. \ref{fig:time_trace}d\&e.


\subsection{Numerics}
We utilize the 
Dedalus\footnote{\url{http://dedalus-project.org/}} 
pseudospectral framework \cite{burns&all2016} to time-evolve  
(\ref{eqn:incompressible})-(\ref{eqn:bouss_energy}) 
using an implicit-explicit (IMEX), third-order, four-step 
Runge-Kutta timestepping scheme RK443 \cite{ascher&all1997}.  
The temperature field is decomposed as $T = T_0(z) + T_1(x, y, z, t)$
and the velocity is $\bm{u} = w\bm{\hat{z}} + u\bm{\hat{x}} + v\bm{\hat{y}}$.
In our 2D runs, $v = 0$.
Variables are time-evolved on a dealiased Chebyshev (vertical)
and Fourier (horizontal, periodic) domain in which the
physical grid dimensions are 3/2 the size of the coefficient grid.  
Domain sizes range from
32x128 coefficients at the lowest values of 
Ra to 1024x4096 coefficients at Ra $> 10^{9}$ in 2D.

As initial conditions, we fill $T_1$ with
random white noise whose magnitude is $10^{-6}(\text{Ra Pr})^{-1/2}$.
This ensures that the initial perturbations are much smaller than the
evolved convective temperature perturbations, even at large Ra.
We filter this noise spectrum in coefficient space, 
such that only the lower 25\% of the coefficients
have power.

In 2D, there are often multiple steady state solutions (e.g., 2-roll and 3-roll
solutions) which have slightly different flow properties (heat transport, etc.).
Even though the initial perturbations are very small, they shape the convective
transient and thus determine the nature of the steady state convection, at least in
the laminar regime.  In order to ensure that our results are not biased by differences
in flow structure, we ran the simulations using distinct random temperature perturbations
so as to compare statistics in comparable flow fields.  In 3D, rolls are nonstationary over
convective timescales, and so these effects need not be considered there.

\section{Results}
While the differences in the fluxes in Fig. \ref{fig:time_trace} are small, it is important
to determine if the velocity fluctuations and point-by-point nonlinear transport are the same
in the evolved state.  Fig. \ref{fig:pdf_comparison} overlays the probability distribution functions
of the vertical and horizontal velocities, as well as the fully nonlinear portion of the convective
flux for the same case as is shown in Fig. \ref{fig:time_trace}.  The PDFs are quite similar visually,
and have a similarity of (XYZ) according to a Kolmogorov-Smirnov statistic.

In addition to getting the nonlinear dynamics mostly correct, we show that the BVP method retrieves the
proper temperature profile, see e.g., Fig. \ref{fig:temp_comparison}.  Here the BVP profile retrieves the
mean profile of the temperature to within 1\% accuracy, and temperature fluctuations in the two runs
have a similarity of (XYZ) according to a Kolmogorov-Smirnov statistic.

This method works across a broad range of supercriticality.  In Fig. \ref{fig:parameter_space_comparison},
we show measurements of the volume-averaged Nusselt number, Reynolds number, and temperature.
We use standard definitions of the Nusselt number and Reynolds numbers,
\begin{equation}
\text{Nu} = \frac{\angles{wT - (\text{Ra Pr})^{-1/2}\grad T}}{\angles{- (\text{Ra Pr})^{-1/2} \grad T}} =
1 + \frac{\angles{wT}}{-\Delta T}\sqrt{\text{Ra Pr}}, \qquad \text{Re} = \frac{|\bm{u}| L_z}{\nu},
\end{equation}
where $\Delta T = T(z = 1/2) - T(z = -1/2)$ is the evolved temperature difference
between the top and bottom plates.  This form of the Nusselt number is valid even when
the system is not yet in flux equilibrium, and reduces to the standard fixed flux definition
of Nu  = $[1 - \angles{wT} / P]^{-1}$ \cite{johnston&doering2009}. We find a scaling law of
Nu $\propto \text{Ra}^{2/3}$, much steeper than a standard 2/7 or 1/3 scaling law
\cite{johnston&doering2009}. Furthermore, we find that Re$\propto \text{Ra}^{0.425}$. The average temperature
approaches the value at the top of the domain as Ra increases.  

The final morphology of the flows is very important in determining the exact value of Nu and Re.  In general,
a two-roll state vs. a three-roll state will have entirely different statistics -- different Nu, Re, and average
temprature (and as a result, different size boundary layers).  Thus, in 2D studies, it is essential to study flows
of a similar morphology in order to see a clear trend. 

\begin{figure}[t]
\includegraphics[width=\textwidth]{./figs/pdf_comparison.png}
\caption{We compare the time-variant dynamics in our 2D solutions by showing probability distribution
functions of the values of vertical velocity (a), horizontal velocity (b), and nonlinear heat
transport (c) in our converged solutions.  The horizontal velocities are very similar, with a
KS test similarity of X.  The vertical velocities, too, are similar, with a KS similarity of
Y.  The nonlinear heat transport, while overall quite similar, shows that the BVP experiences
a greater number of extreme events.  The peak around zero for all three PDFs reflects the
no-slip boundary conditions.\label{fig:pdf_comparison} }
\end{figure}

\begin{figure}[t]
\includegraphics[width=\textwidth]{./figs/temp_comparison.png}
\caption{Here we compare the evolved thermodynamic state of a BVP to that of a rundown IVP that
has been run through a thermal timescale.  (a) The temperature profiles, as a function of height,
are shown. (b) The percentage difference between the temperature profiles, as a function of height,
is shown.  (c) PDFs comparing the values of the point-by-point temperatures over the averaging
window.  While the median of the PDF has a distinct offset between the two runs, the
thermal fluctuations about the mean which drive the convection are nearly the same.
\label{fig:temp_comparison} }
\end{figure}

\begin{figure}[t]
\includegraphics[width=\textwidth]{./figs/parameter_space_comparison.png}
\caption{Here we demonstrate the efficacy of the BVP method at many values of Ra.  We show
(a) Nusselt number scaling, (b) Reynolds number scaling, and (c) scaling of the mean temperature
over the whole simulation.  (a) The Nusselt number follows a classic 2/7 scaling law
\cite{johnston&doering2009}, until it reaches S = $10^{3 + 2/3}$, at which point the
two-roll solution begins oscillating horizontally, showing time-variant Nu.  The range of Nu
over the averaging window is (blah), and the trend of Nu (the variance of the mean Nu) is shown
as (blah).  (b) Re follows an expected 1/2 scaling law. (c) The mean value of temperature, minus
the value at the fixed-temperature boundary, is shown.  As Ra increases, the boundary layers shrink,
and the convection becomes more efficient, we anticipate this value to get closer and closer to
zero as the atmosphere approaches the isotherm defined by the upper boundary.
\label{fig:parameter_space_comparison} }
\end{figure}



\section{Discussion \& Conclusions}
\label{sec:results}
While imperfect, the method presented here is a first step towards taking meaningful measurements
of highly turbulent convection on manageable, human timescales.  As demonstrated in Figs.
1-4, this BVP method quickly converges simulations to within a few percent of the true final
state, while preserving the natural behavior of the convective solution (such as the oscillatory
nature of high-Ra 2D states in \ref{fig:parameter_space_comparison}).  

While not perfect, the BVP method here has one major benefit over some of the other methods
currently being used to achieve high Ra solutions.  One oft-used method is that of bootstrapping,
in which the converged solution of a low-Ra state is used as initial conditions for a higher-Ra
simulation.  While these methods are extremely powerful, they are influenced by hysteresis effects,
and the powerful, steady rolls achieved at low Ra can result in an artifically over-stable high-Ra
roll solution.  The BVP method can be used with random noise initial conditions which allow the
convective solution to be naturally chosen by the dynamics.

One benefit of the method presented here is that it can be easily extended to more complicated
configurations.  For example, to use this method in simulations of stratified compressible convection,
one need only adapt the BVP equations to the appropriate equations of hydrostatic equilibrium,
$\grad P = -\rho \bm{g}$, and thermal equilibrium, $\Div{F_{\text{cond}} + F_{\text{conv}}} = \text{sources}$,
for the problem at hand.  In compressible convection, where the density is allowed to change,
one must also use the knowledge of stellar structure codes and add an equation of mass conservation
in order to ensure that the BVP does not spuriously add or remove mass from the system.

This method can be extended to other boundary conditions, as well.  To solve for fixed temperature
boundary conditions, the difficulty is in finding the amount of flux through the system -- but
this can be done by using the ratios $f_{\text{conv}}$ and $f_{\text{cond}}$.  In the case of
fixed flux boundary conditions, there is degeneracy in the temperature solution which can come
out of the BVP, but in using knowledge about the system -- such as the initial symmetry of the
RB state around $T_1 = 0$, the final solution can be pegged onto the proper profile.




\begin{acknowledgments}
EHA acknowledges the support of the University of Colorado's George 
Ellery Hale Graduate Student Fellowship.
This work was additionally supported by  NASA LWS grant number NNX16AC92G.  
Computations were conducted 
with support by the NASA High End Computing (HEC) Program through the NASA 
Advanced Supercomputing (NAS) Division at Ames Research Center on Pleiades
with allocations GID s1647 and GID g26133.
\end{acknowledgments}


\appendix
\section{Table of Runs}
\begin{center}
\begin{tabularx}{\textwidth}{ X X X X X X X }
\hline													
Ra	&	Supercriticality	&	nz	&	nx	&	$t_{\text{therm}}$	&	$t_{\text{post BVP}}$	&	$t_{\text{avg}}$	\\[1ex]
\hline		\hline											
$2.79 \cdot 10^3$	&	$10^{1/3}$	&	32	&	128	&	$52.8$	&	50	&	100	\\
$6.01 \cdot 10^3$	&	$10^{2/3}$	&	32	&	128	&	$77.6$	&	50	&	100	\\
$1.30 \cdot 10^4$	&	$10^1$	&	32	&	128	&	$114$	&	50	&	100	\\
$2.79 \cdot 10^4$	&	$10^{1 + 1/3}$	&	32	&	128	&	$167$	&	50	&	100	\\
$6.01 \cdot 10^4$	&	$10^{1 + 2/3}$	&	32	&	128	&	$245$	&	50	&	100	\\
$1.30 \cdot 10^5$	&	$10^2$	&	64	&	256	&	$360$	&	100	&	100	\\
$2.79 \cdot 10^5$	&	$10^{2 + 1/3}$	&	64	&	256	&	$528$	&	100	&	100	\\
$6.01 \cdot 10^5$	&	$10^{2 + 2/3}$	&	64	&	256	&	$776$	&	100	&	100	\\
$1.30 \cdot 10^6$	&	$10^3$	&	128	&	512	&	$1.14 \cdot 10^3$	&	100	&	200	\\
$2.79 \cdot 10^6$	&	$10^{3 + 1/3}$	&	128	&	512	&	$1.67 \cdot 10^3$	&	200	&	200	\\
$6.01 \cdot 10^6$	&	$10^{3 + 2/3}$	&	256	&	1024	&	$2.45 \cdot 10^3$	&	200	&	200	\\
$1.30 \cdot 10^7$	&	$10^4$	&	256	&	1024	&	$3.60 \cdot 10^3$	&	200	&	200	\\
$2.79 \cdot 10^7$	&	$10^{4 + 1/3}$	&	256	&	1024	&	$5.28 \cdot 10^3$	&	200	&	200	\\
$6.01 \cdot 10^7$	&	$10^{4 + 2/3}$	&	256	&	1024	&	$7.76 \cdot 10^3$	&	200	&	200	\\
$1.30 \cdot 10^8$	&	$10^5$	&	512	&	2048	&	$1.14 \cdot 10^4$	&	500	&	500	\\
$2.79 \cdot 10^8$	&	$10^{5 + 1/3}$	&	512	&	2048	&	$1.67 \cdot 10^4$	&	500	&	500	\\
$6.01 \cdot 10^8$	&	$10^{5 + 2/3}$	&	512	&	2048	&	$2.45 \cdot 10^4$	&	500	&	500	\\
$1.30 \cdot 10^9$	&	$10^6$	&	1024	&	4096	&	$3.60 \cdot 10^4$	&	500	&	500	\\
\hline													
\end{tabularx}
\end{center}



\bibliography{biblio.bib}
\end{document}
