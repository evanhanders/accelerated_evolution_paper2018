%\documentclass[iop]{emulateapj}
\documentclass[aps, pre, onecolumn, nofootinbib, notitlepage, groupedaddress, amsfonts, amssymb, amsmath, longbibliography]{revtex4-1}
\usepackage{graphicx}
\usepackage{hyperref}
\usepackage{xcolor}
\hypersetup{
    colorlinks,
    linkcolor={red!50!black},
    citecolor={blue!50!black},
    urlcolor={blue!80!black}
}
\usepackage{bm}
\usepackage{natbib}
\usepackage{longtable}
\LTcapwidth=0.87\textwidth

\newcommand{\Div}[1]{\ensuremath{\nabla\cdot\left( #1\right)}}
\newcommand{\angles}[1]{\ensuremath{\left\langle #1 \right\rangle}}
\newcommand{\grad}{\ensuremath{\nabla}}
\newcommand{\RB}{Rayleigh-B\'{e}nard }
\newcommand{\stressT}{\ensuremath{\bm{\bar{\bar{\Pi}}}}}
\newcommand{\lilstressT}{\ensuremath{\bm{\bar{\bar{\sigma}}}}}
\newcommand{\nrho}{\ensuremath{n_{\rho}}}
\newcommand{\approptoinn}[2]{\mathrel{\vcenter{
	\offinterlineskip\halign{\hfil$##$\cr
	#1\propto\cr\noalign{\kern2pt}#1\sim\cr\noalign{\kern-2pt}}}}}

\newcommand{\appropto}{\mathpalette\approptoinn\relax}

\newcommand\mnras{{MNRAS}}%

\begin{document}
\author{Evan H. Anders}
\affiliation{Dept. Astrophysical \& Planetary Sciences, University of Colorado -- Boulder, Boulder, CO 80309, USA}
\affiliation{Laboratory for Atmospheric and Space Physics, Boulder, CO 80303, USA}
\author{Benjamin P. Brown}
\affiliation{Dept. Astrophysical \& Planetary Sciences, University of Colorado -- Boulder, Boulder, CO 80309, USA}
\affiliation{Laboratory for Atmospheric and Space Physics, Boulder, CO 80303, USA}
\author{Jeffrey Oishi}
\affiliation{Bates}
\title{BVP Paper}

\begin{abstract}
WOW this is a really long sentence check out this abstract I'll just keep writing words to make this at least
one line long so we know what the formatting looks like, ok?
\end{abstract}
\maketitle


\section{Introduction}
\label{sec:intro}
In many simple studies of convection, the choice of boundary conditions greatly influences the dynamics
of the evolved state.  In studies of incompressible, Boussinesq, \RB convection (RBC), many studies
employ fixed temperature (Dirichlet) boundary conditions at the top and bottom, or fixed heat flux
(Neumann) boundary conditions at both plates.  The former case represents plates of infinite conductivity,
whereas the latter plates of finite conductivity.  In both cases, the choice of symmetric boundary
conditions maintains overall system symettry, and both have been shown to transport heat in nearly the
same manner \cite{johnston&doering2009}.

Studies of convection in stratified systems which aim to model convection in natural, astrophysical systems,
such as that in the outer envelope of low-to-moderate mass stars like the Sun, often employ a mixture of these
two types of boundary conditions (references, references).  The flux at the lower boundary is fixed, modeling
the constant energy generation of the stellar core, while the outer boundary condition is held at a fixed temperature,
effectively allowing all heat generated to escape.  

While this setup is a useful model for understanding natural
systems, simulations which employ this setup often suffer from a long convective transient as the thermodynamic structure
of the atmosphere relaxes to the adiabatic profile specified by the fixed temperature upper boundary condition.
This long evolution occurs on the ``Kelvin-Helmholtz,'' or thermal diffusion timescale of the atmosphere, 
$t_{\text{KH}} \approx L_z^2 / \chi$, where $L_z$ is the domain depth and $\chi$ is the thermal diffusivity.
Interesting convection happens as high values of the Rayleigh number, which scales like $\chi^{-1/2}$, such that
in the astrophysically interesting regime of high-Ra, highly stratified convection, evolving a simulation for a
KH time becomes numerically intractable.  As the Rayleigh number increases, the KH time increases while the average
timestep required to resolve the more turbulent flows decreases.  The net result is that under standard initial conditions
of hydrostatic- and thermal- equilibrium, the desired convective solution cannot be obtained and the dynamics of convection
there cannot be studied.

Here we present a method for using simple boundary value problems, along with information about the evolved flow fields,
to fast-forward the slow thermal evolution of convecting simulations.  We present this method in the context of RBC, and
then demonstrate that it applies to stratified, compressible convection under a simple modification.  These methods allow
us to study the convective flows driven by the evolved thermal profile while only requiring initial value problems to run
for long enough to resolve the fast dynamical timescales of convection.


\section{Experiment} 
\label{sec:experiment}
Here we talk about how to set up the RB problem that we're going to solve, and exactly what
we're going to compare (how long the run is going to go from base and from bvp, how much time
we're going to compare over after that, etc.)  We need to compare over a short enough timescale
that the effects of BVPs don't get washed away, and we also need to compare over long enough
timescales that we get the rare events we expect.  This is where we put a figure showing the
time evolution of energy in IVP / BVP side-by-side.

\section{Results \& Discussion}
\label{sec:results}
Here we talk about how the solutions are different, or similar.  This includes:
\begin{enumerate}
\item Showing that the flow fields look similar
\item Showing how the temperature / flux profiles look similar/different
\item showing how Nu and Re scale with Ra in BVP / IVP.
\item showing how the PDFs of $w$, $wT$, and $T$ change.
\end{enumerate}

Then we need to make some comments about whether this is good or bad

Then we need to mention how the same thing can be done in stratified, just there you don't
assume symmetrical boundary layers.





\begin{acknowledgments}
EHA acknowledges the support of the University of Colorado's George 
Ellery Hale Graduate Student Fellowship.
This work was additionally supported by  NASA LWS grant number NNX16AC92G.  
Computations were conducted 
with support by the NASA High End Computing (HEC) Program through the NASA 
Advanced Supercomputing (NAS) Division at Ames Research Center on Pleiades
with allocations GID s1647 and GID g26133.
\end{acknowledgments}


\appendix
\section{Process for setting enthalpy flux}


\bibliography{biblio.bib}
\end{document}
