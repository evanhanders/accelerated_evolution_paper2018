%\documentclass[iop]{emulateapj}
\documentclass[aps, pre, onecolumn, nofootinbib, notitlepage, groupedaddress, amsfonts, amssymb, amsmath, longbibliography]{revtex4-1}
\usepackage{graphicx}
\usepackage{hyperref}
\usepackage{xcolor}
\hypersetup{
    colorlinks,
    linkcolor={red!50!black},
    citecolor={blue!50!black},
    urlcolor={blue!80!black}
}
\usepackage{bm}
\usepackage{natbib}
\usepackage{longtable}
\LTcapwidth=0.87\textwidth

\newcommand{\Div}[1]{\ensuremath{\nabla\cdot\left( #1\right)}}
\newcommand{\angles}[1]{\ensuremath{\left\langle #1 \right\rangle}}
\newcommand{\grad}{\ensuremath{\nabla}}
\newcommand{\RB}{Rayleigh-B\'{e}nard }
\newcommand{\stressT}{\ensuremath{\bm{\bar{\bar{\Pi}}}}}
\newcommand{\lilstressT}{\ensuremath{\bm{\bar{\bar{\sigma}}}}}
\newcommand{\nrho}{\ensuremath{n_{\rho}}}
\newcommand{\approptoinn}[2]{\mathrel{\vcenter{
	\offinterlineskip\halign{\hfil$##$\cr
	#1\propto\cr\noalign{\kern2pt}#1\sim\cr\noalign{\kern-2pt}}}}}

\newcommand{\appropto}{\mathpalette\approptoinn\relax}

\newcommand\mnras{{MNRAS}}%

\begin{document}
\title{BVP Methods}

\maketitle


\section{Brief description of method}
In a really simple sense, this is the procedure that is being done when I do the
``BVP solve'' for getting a more converged atmosphere.
\begin{enumerate}
\item Start up IVP, and run it.
\item Once the flows hit Re = 1, wait some time, $t_{\text{transient}}$.
\item After $t_{\text{transient}}$, start taking horizontal and time averages of
specified profiles in the atmosphere.
\item Once the profiles are converged (the change in the profiles at the next timestep
change, on average, less than $f$, where $f$ is a fraction.  If $f=0.01$, the profiles
change no more than 1\% from the previous timestep.), feed them into a 1D problem. 
\item In the 1D problem, adjust the profiles from the atmosphere intelligently.  Feed
those profiles into a BVP that obeys the same thermal boundary conditions as the IVP,
then adjust the mean thermal profile of the IVP atmosphere with the result of the BVP.
\item Keep running the IVP.  If desired, wait some time, $t_{\text{equil}}$, and then restart
from step 3.

\section{More thorough write-up, Boussinesq convection}
In boussinesq convection, the time-steady energy equation can be written as
$$
\Div{ \bm{u} T - \kappa \grad T} = 0
$$
When we take horizontal averages and time averages, and assume constant $\kappa$,
this equation reduces to
$$
\partial_z \angles{(w T)} - \kappa \partial^2_z \angles{(T_0 + T_1)} = 0,
$$
where angles represent a time- and horizontal average.  Further, due to symmetry,
when we take a time- and horizontal average, we find that most of the terms in the 
Rayleigh-Benard momentum equation go away.  Even if they hadn't, the main thing we want
from the momentum equation is an update to the hydrostatic balance of the atmosphere, and
that means that we need to update
$$
\partial_z \angles{p} = \angles{T_1}.
$$

So, put simply, the BVP that we need to solve in RBC is 
\begin{equation}
\begin{split}
\frac{\partial T_1}{\partial z} - T_{1z} &= 0 \\
\kappa \frac{\partial T_{1z}}{\partial z} &= \frac{\partial}{\partial z} \angles{w T}  - \kappa \frac{\partial^2 T_0}{\partial z^2}\\
\frac{\partial p}{\partial z} - T_1 &= 0.
\end{split}
\end{equation}
These three equations, coupled with two properl thermal boundary conditions, retrieve the full
thermodynamic state of the atmosphere.  I've written the equations in their dedalus-like form,
for clarity.

In a real way, the profile of the enthalpy flux thus sets the profile of the thermal structure of
the atmosphere.



\end{enumerate}
\end{document}
