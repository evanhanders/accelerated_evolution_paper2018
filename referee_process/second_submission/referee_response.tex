%----------------------------------------------------------------------------------------
%	PACKAGES AND OTHER DOCUMENT CONFIGURATIONS
%----------------------------------------------------------------------------------------

\documentclass[aps, 11pt, singlecolumn]{revtex4-1} % Set the font size (10pt, 11pt and 12pt) and paper size (letterpaper, a4paper, etc)
\usepackage{natbib}
\bibliographystyle{apsrev}
\usepackage{setspace}

\begin{document}
\newenvironment{myquotation}{
\begin{quotation}
\itshape
}{ 
\end{quotation}
}
%----------------------------------------------------------------------------------------
%	LETTER CONTENT
%----------------------------------------------------------------------------------------
\noindent
Dear Eckart, other PRFluids editors, and anonymous referees,
$\,$\newline

\begin{singlespace}
We return our revised manuscript, ``Accelerated evolution in convective simulations,''
clarifying the paper where requested by the report of referee 1. 
Aside from these changes, the 
final pargaraph of the introductory section of the 
manuscript now contains references to the works of
Julien et al 1998 (ref 16) and Sprague et al 2006 (ref 17), whose studies of
rapidly rotating, asymptotically reduced convection 
we felt should be mentioned in this work.

We appreciate the very careful work of the referees in improving the clarity of
this manuscript. Below ew include our responses to the report; the original
text of the report is included in italics, and our response follows inline in
unitalicized blocks.

$\,$\newline
\noindent
Sincerely,

Evan H. Anders, Benjamin P. Brown, and Jeffrey S. Oishi



%\end{singlespace}

\newpage
\noindent
\Large{\textbf{Response to first referee:}}\newline$\,$\newline\indent

\begin{myquotation}
I am ok with the paper being published now.   There are a small number of things
that could be done that would still help the clarity of the paper:

Intro:  This is perhaps a bit nit-picky, but the authors talk about
“equilibration times” whereas perhaps they should be more precise and refer to
what they really mean, i.e. the thermal relaxation time.   There is no natural
timescale that is the equilibration time, since this depends on where you start
from.  One of the points made in the intro is that one way around the thermal
relaxation problem is to start near the true solution by “bootstrapping”, but
this requires a guess at the nonlinear solution which is hard in a turbulent
system.  So the problem really arises because, without such a guess, the
simulation follows the full evolution from unstable equilibrium to stable
equilibrium which have very different thermal profiles.  The equilibration time
is therefore just the thermal relaxation time, which is one of the many
timescales that the authors refer too.  I think it is better to just talk in
terms of widely separated timescales (a stiff problem) and be specific that the
one they are interested in is the stiffness between convective and thermal
relaxation, i.e. high Pe problems.
\end{myquotation}
This is a good (and subtle) point. Where appropriate, we have changed the text
to refer to thermal relaxation rather than thermal equilibration throughout.

\begin{myquotation}
In referring to reference [16], I think it is more accurate to say that this is
one EXAMPLE of where this technique has been used MANY times before.
\end{myquotation}
Ref. 16 (which is now ref. [18], due to the addition of the sentence before it
which calls out new refs [16, 17]) is now appropriately called out as an example
of where a similar technique has been used before.

\begin{myquotation}
Sect III para 2:  ``proper'' -- I think you mean ``nonlinear'' or ``saturated''
\end{myquotation}
Correct! ``proper'' $\rightarrow$ ``saturated''.

\begin{myquotation}
Sect IV para 2: ``oscillatory nature is stable''  -- I am not sure what this means!
\end{myquotation}
The plumes oscillate for long timescales (the oscillatory motions do not
eventually give way to stationary plumes). We have modified this sentence to
better explain our meaning.

\begin{myquotation}
My personal opinion is that I still think the section about the scaling laws is
weird to have here, in the sense that it interrupts the flow of the paper.  But
it is not wrong, so I guess it can stay!

Sect VI:  I think this section has improved in the more cautionary and less
speculative way as suggested.  I would add a comment to the penultimate line
that says that there could be issues with sound waves in compressible
calculations which are very sensitive to pressure mismatches.
\end{myquotation}
A sentence explaining this point has been added where suggested.

\end{singlespace}




\bibliography{../../biblio.bib}
\end{document}
